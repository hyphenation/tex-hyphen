% xu-grphyph4.tex
% Wrapper to read grphyph4.tex (TeX) or utf8-grphyph4.tex (XeTeX)
% Jonathan Kew, 2006-08-19
% Public domain

\begingroup

\expandafter\ifx\csname XeTeXrevision\endcsname\relax

  \input grphyph4.tex

\else

  \input utf8-grphyph4.tex

\fi

\endgroup
\endinput
